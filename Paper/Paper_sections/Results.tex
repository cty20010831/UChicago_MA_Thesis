\documentclass[../MA_Thesis.tex]{subfiles}
\renewcommand{\baselinestretch}{1.5} 
\usepackage{hyperref}
\usepackage{csquotes}
\usepackage{float}
\usepackage{threeparttable}  % for table notes
\usepackage{booktabs}        % for \toprule, \midrule, \bottomrule

\begin{document}
\subsection*{Participant Demographics}

Of the final sample (N = 90), participants ranged in age from 19 to 65 years ($M = 33.5$, $SD = 10.4$). The sample included 43 females (47.8\%), 44 males (48.9\%), and 3 participants identifying as non-binary or other (3.3\%). 47.8\% identified as White, 13.3\% as Black or African American, 13.3\% as Asian, 12.2\% as Hispanic or Latino, and 13.3\% as multiracial or other. Educational attainment was relatively high: 42.2\% of participants held a bachelor’s degree, and 24.4\% had completed a graduate degree. All participants were physically located in the United States.

\subsection*{Manipulation Check: Mood Induction}
To verify that the mood induction procedure successfully manipulated participants’ mood states along both the arousal and valence dimensions, I conducted one-way ANOVAs with mood group (\textit{High-Arousal Positive Mood}, \textit{Low-Arousal Positive Mood}, and
\textit{Neutral Control}) as the between-subjects factor.

For arousal ratings, a one-way ANOVA revealed a significant effect of induced mood condition, $F(2, 87) = 6.86$, $p = .0017$, with a moderate effect size ($\eta^2 = .14$). Furthermore, post-hoc Tukey tests showed that participants in the \textit{High-Arousal Positive Mood} group ($M = 6.70$, $SD = 1.02$) reported significantly higher arousal than those in the \textit{Neutral Control} group ($M = 4.05$, $SD = 1.12$; $p = .0021$) and the \textit{Low-Arousal Positive Mood} group ($M = 4.53$, $SD = 1.09$; $p = .0171$). No significant difference was observed between the \textit{Neutral Control} and \textit{Low-Arousal Positive Mood} groups ($p = .7641$; see Figure~\ref{fig:arousal_group}). 

\begin{figure}[H]
  \centering
  \includegraphics[width=\textwidth]{../analysis/results/main_results/mood_induction_check/arousal_by_group.png}
  \caption{Boxplot of Arousal Ratings by Mood Group}
  \label{fig:arousal_group}
\end{figure}

For valence ratings, a one-way ANOVA also showed a significant group difference, $F(2, 87) = 4.44$, $p = .0146$, with a moderate effect size ($\eta^2 = .09$). Moreover, post-hoc Tukey tests revealed that the \textit{High-Arousal Positive Mood} group ($M = 2.91$, $SD = 0.96$) reported significantly higher valence than the \textit{Neutral Control} group ($M = 1.24$, $SD = 1.24$; $p = .0167$), whereas the \textit{Low-Arousal Positive Mood} group ($M = 2.16$, $SD = 1.10$) did not significantly differ from either group ($ps > .05$; see Figure~\ref{fig:valence_group}). 

\begin{figure}[H]
  \centering
  \includegraphics[width=\textwidth]{../analysis/results/main_results/mood_induction_check/valence_by_group.png}
  \caption{Boxplot of Valence Ratings by Mood Group}
  \label{fig:valence_group}
\end{figure}

Overall, the experimental mood induction procedure using three validated film clips achieved the anticipated mood manipulation results. For the two positive valence groups, there was a significant difference in arousal ratings, whereas the valence ratings did not significantly differ. This pattern supports the intended orthogonal manipulation of arousal while holding valence constant, allowing for a more focused investigation of arousal's unique effects on subsequent creative processes. Meanwhile, the significant differences in both arousal and valence ratings between the \textit{High-Arousal Positive Mood} and \textit{Neutral Control} groups confirm the successful induction of a distinctly activated positive affective state in the high-arousal condition.

\subsection*{Descriptive Statistics of Flexibility and Originality Measures}

The descriptive statistics for all the measures of flexibility and originality across mood conditions are presented in Table~\ref{tab:descriptive_stats}. Overall, the flexibility metrics demonstrated subtle group differences. Average entropy and Bhattacharyya distance were highest in \textit{High-Arousal Positive} group and lowest in \textit{Low-Arousal Positive} group, though these differences were small. Inflection proportions of entropy and Bhattacharyya distance were slightly elevated in \textit{Low-Arousal Positive} group, suggesting greater within-trial shifts in drawing strategy. DSI scores were highest in \textit{High- Arousal Positive} group, indicating more semantically diverse narrative content, with \textit{Neutral Control} and \textit{Low-Arousal Positive} groups showing similar levels. Originality, as measured by the AuDrA model, remained stable under all conditions, with means ranging narrowly from 0.36 to 0.37 and no apparent variation across groups. 

\begin{table}[H]
\centering
\begin{threeparttable}
\caption{Descriptive Statistics by Mood Condition (Mean (SD))}
\label{tab:descriptive_stats}
\begin{tabular}{lccc}
\toprule
\textbf{Measure} & \textbf{Neutral Group} & \textbf{High Arousal Group} & \textbf{Low Arousal Group} \\
\midrule
Avg. Entropy & 1.77 (0.09) & 1.78 (0.07) & 1.74 (0.10) \\
Avg. Bhatt. Dist. & 2.51 (0.29) & 2.50 (0.25) & 2.45 (0.28) \\
Inflect. Prop. Entropy & 0.41 (0.18) & 0.42 (0.17) & 0.45 (0.17) \\
Inflect. Prop. Bhatt & 0.42 (0.18) & 0.43 (0.16) & 0.44 (0.20) \\
DSI & 0.59 (0.28) & 0.66 (0.22) & 0.58 (0.30) \\
AuDrA & 0.37 (0.10) & 0.36 (0.10) & 0.37 (0.11) \\
\bottomrule
\end{tabular}
\begin{tablenotes}[flushleft]
\small
\item \textit{Note.} Avg. Entropy = Average Entropy; Avg. Bhatt. Dist. = Average Bhattacharyya Distance; Inflect. Prop. Entropy = Inflection Proportion of Entropy; Inflect. Prop. Bhatt = Inflection Proportion of Bhattacharyya Distance; DSI = Divergent Semantic Integration; AuDrA = Automated Drawing Assessment (Originality Score).
\end{tablenotes}
\end{threeparttable}
\end{table}

\subsection*{Correlational and Predictive Analyses of Flexibility Measures}

Correlation analyses revealed a notable divergence between two modes of cognitive flexibility. As shown in Figure~\ref{fig:cor_heatmap}, average entropy and average Bhattacharyya distance were strongly positively correlated with each other ($r = .83$), but both showed weak negative correlations with their respective inflection metrics (e.g., $r = -0.21$ for entropy and entropy inflection proportion) and with DSI ($r$s between –.12 and –.18). These trends suggest that participants who sustained a broad space of exploratory stroke options throughout the drawing process tended to make fewer within-trial shifts in drawing strategy and produced less semantically divergent narratives.

\begin{figure}[H]
  \centering
  \includegraphics[width=\textwidth]{../analysis/results/main_results/correlation/correlation_heatmap_flexibility.png}
  \caption{Pearson Correlation Matrix Among Flexibility Measures}
  \label{fig:cor_heatmap}
\end{figure}

By contrast, the two inflection-based measures were strongly positively correlated with each other ($r = .88$), and both were moderately positively associated with DSI ($r > .50$), indicating a shared flexibility dynamic rooted in frequent strategic switching and broader conceptual integration. As illustrated in Figure~\ref{fig:pairwise_flexibility_originality}, these switching-based flexibility metrics demonstrated a modest positive relationship with originality (AuDrA), while the average metrics (entropy and Bhattacharyya distance) showed little to no association with originality.

\begin{figure}[H]
  \centering
  \includegraphics[width=\textwidth]{../analysis/results/main_results/correlation/ggpairs_flexibility_and_outcome.png}
  \caption{Pairwise Relationships Among Flexibility Metrics and Originality (AuDrA)}
  \label{fig:pairwise_flexibility_originality}
\end{figure}

Taken together, these results suggest that real-time flexibility may manifest through two distinct behavioral strategies: one characterized by persistent exploratory breadth (high average entropy/Bhattacharyya), and another characterized by dynamic switching between ideas (high inflection proportions and DSI). Importantly, only the latter appears meaningfully connected to originality in this task context.

\subsection*{Mediation Analysis}

To evaluate whether mood condition influenced creative originality through different aspects of cognitive flexibility, we conducted a series of mediation models, testing each flexibility measure individually as a mediator. Mood conditions were dummy-coded using the Neutral Control group as the reference category. The outcome variable in all models was originality, measured via AuDrA.

\paragraph{Average Entropy.}
Mood condition had no significant effect on average entropy (D1: $\beta = -0.010$, $p = .646$; D2: $\beta = -0.046$, $p = .039$), and average entropy did not significantly predict originality ($\beta = -0.255$, $p = .010$). While the path b effect was statistically significant, the indirect effects were not: indirect$_{D1}$ = 0.003 ($p = .672$), indirect$_{D2}$ = 0.012 ($p = .106$). Direct effects of mood on originality were non-significant, and total effects were close to zero.

\paragraph{Average Bhattacharyya Distance.}
Neither mood condition (D1: $\beta = -0.042$, $p = .552$; D2: $\beta = -0.107$, $p = .113$) nor average Bhattacharyya distance ($\beta = -0.037$, $p = .239$) significantly predicted originality. Indirect effects were also non-significant (indirect$_{D1}$ = 0.002, $p = .658$; indirect$_{D2}$ = 0.004, $p = .359$), and direct effects remained negligible.

\paragraph{Inflection Proportion of Entropy.}
Mood condition had no significant effect on inflection proportion (D1: $\beta = 0.018$, $p = .649$; D2: $\beta = 0.042$, $p = .268$), but inflection proportion strongly predicted originality ($\beta = 0.447$, $p < .001$). Indirect effects were not significant (indirect$_{D1}$ = 0.008, $p = .652$; indirect$_{D2}$ = 0.019, $p = .266$), but the strength of the b-path suggests this mediator may still play a functional role. Direct effects remained non-significant.

\paragraph{Inflection Proportion of Bhattacharyya Distance.}
Patterns were similar: no significant effect of mood condition on the mediator (D1: $\beta = 0.024$, $p = .540$; D2: $\beta = 0.027$, $p = .510$), but a strong effect of inflection proportion on originality ($\beta = 0.446$, $p < .001$). Indirect effects again were non-significant (indirect$_{D1}$ = 0.011, $p = .542$; indirect$_{D2}$ = 0.012, $p = .511$).

\paragraph{Divergent Semantic Integration (DSI).}
DSI was weakly predicted by mood condition (D1: $\beta = 0.078$, $p = .163$; D2: $\beta = 0.008$, $p = .897$), but it significantly predicted originality ($\beta = 0.219$, $p < .001$). Although the indirect effects were not statistically significant (indirect$_{D1}$ = 0.017, $p = .177$; indirect$_{D2}$ = 0.002, $p = .898$), the direct effects of mood on originality were also non-significant, suggesting a partial mediation.

Overall, while mood induction did not significantly influence most flexibility measures, originality was reliably predicted by DSI and inflection-based flexibility. These findings support the notion that originality may depend more on dynamic, adaptive shifts and conceptual integration rather than on sustained exploratory breadth alone.

\begin{table}[H]
{\fontsize{9pt}{11pt}\selectfont
\centering
\caption{Mediation Model Summary: Flexibility Measures as Mediators of Mood on Originality}
\label{tab:mediation_summary}
\begin{tabular}{lcccc}
\toprule
\textbf{Mediator} & \textbf{Path a} ($\beta$) & \textbf{Path b} ($\beta$) & \textbf{Indirect Effect} & \textbf{Direct Effect} \\
\midrule
Avg. Entropy & D1: –0.010 ($p = .646$) & –0.255*** ($p = .010$) & D1: 0.003 [–.010, .015] ($p = .672$) & D1: –0.006 ($p = .759$) \\
             & D2: –0.046* ($p = .039$) &                         & D2: 0.012 [–.000, .028] ($p = .106$) & D2: –0.005 ($p = .821$) \\
\addlinespace
Avg. Bhatt. Dist. & D1: –0.042 ($p = .552$) & –0.037 ($p = .239$) & D1: 0.002 [–.005, .009] ($p = .658$) & D1: –0.005 ($p = .803$) \\
                  & D2: –0.107 ($p = .113$) &                     & D2: 0.004 [–.003, .015] ($p = .359$) & D2:  0.003 ($p = .899$) \\
\addlinespace
Inflect. Prop. Entropy & D1:  0.018 ($p = .649$) &  0.447*** ($p < .001$) & D1: 0.008 [–.027, .043] ($p = .652$) & D1: –0.011 ($p = .326$) \\
                       & D2:  0.042 ($p = .268$) &                         & D2: 0.019 [–.015, .052] ($p = .266$) & D2: –0.012 ($p = .285$) \\
\addlinespace
Inflect. Prop. Bhatt & D1:  0.024 ($p = .540$) &  0.446*** ($p < .001$) & D1: 0.011 [–.024, .046] ($p = .542$) & D1: –0.014 ($p = .162$) \\
                     & D2:  0.027 ($p = .510$) &                        & D2: 0.012 [–.025, .047] ($p = .511$) & D2: –0.005 ($p = .609$) \\
\addlinespace
DSI & D1: 0.078 ($p = .163$) & 0.219*** ($p < .001$) & D1: 0.017 [–.006, .045] ($p = .177$) & D1: –0.021 ($p = .213$) \\
    & D2: 0.008 ($p = .897$) &                        & D2: 0.002 [–.023, .028] ($p = .898$) & D2:  0.005 ($p = .749$) \\
\bottomrule
\end{tabular}
\begin{tablenotes}[flushleft]
\small
\item \textit{Note.} * $p < .05$, ** $p < .01$, *** $p < .001$. Path a = Mood group $\rightarrow$ Flexibility; Path b = Flexibility $\rightarrow$ Originality (AuDrA); Indirect effect = $\text{a} \times \text{b}$; D1 = High Arousal vs. Neutral; D2 = Low Arousal vs. Neutral. Confidence intervals based on 5,000 bootstrapped samples.
\end{tablenotes}
}
\end{table}
\end{document}