\documentclass[../MA_Thesis.tex]{subfiles}
\renewcommand{\baselinestretch}{1.5} 
\usepackage{hyperref}
\usepackage{csquotes}

\begin{document}
This study examined the mood-creativity linkage proposed by the dual pathway to creativity model: Positive activating moods enhance cognitive flexibility, thereby increasing originality in creative output. Specifically, it leveraged the incompleteness drawing task coupled with state-of-the-art NLP and computer vision techniques to derive flexibility measures from the stoke dynamics and post-drawing narratives and originality measures from the final completed drawings. Although mood induction successfully manipulated the arousal and valence scores of \textit{High-Arousal Positive Mood}, \textit{Low-Arousal Positive Mood}, and \textit{Neutral Control} groups, it surprisingly did not yield significant differences in flexibility or originality measures across groups, which could be attributed to a statistically underpowered sample size to capture the subtler effects of mood on flexibility and originality or the transient nature of mood states induced through brief film clips. Nonetheless, the findings revealed two distinct modes of cognitive flexibility: persistent exploratory breadth, reflected by higher average entropy and Bhattacharyya distance in stroke predictions, and adaptive switching, captured by proportion of inflection points in the time series of entropy and Bhattacharyya distance, as well as semantic integration in narratives. Crucially, only adaptive flexibility was significantly associated with originality, supporting the idea that the capacity to change strategies and integrate distinct concepts in real time is central to creative performance. These findings not only highlight the caveats of experimental mood induction for creativity research but also underscore the value of combining visual creativity tasks with computational modeling to unpack the cognitive processes underlying originality. By capturing both stroke-based and narrative expressions of flexibility, this study offers a cross-modal, process-oriented framework for understanding creative performance beyond traditional verbal tasks. 
\end{document}