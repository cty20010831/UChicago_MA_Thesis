\documentclass[../MA_Thesis.tex]{subfiles}
\renewcommand{\baselinestretch}{1.5} 
\usepackage{hyperref}
\usepackage{csquotes}

\begin{document}
This proposed study aims to join the debate on mood-creativity linkage by integrating visual creative tasks with state-of-the-art computational techniques. In line with the flexibility pathway in the dual pathway to creativity model, this study hypothesizes that positive activating moods enhance cognitive flexibility, thereby increasing originality in creative output. Diverging from the predominance of verbal tasks to measure creativity, this study employs an incompleteness drawing task within the Multi-Trial Creative Ideation (MTCI) framework to track the dynamics of the creative process. A custom-built website will be built to induce mood states, administer drawing tasks, and gather narrative accounts of the creative process from approximately 100 participants sourced through Amazon Mechanical Turk. 

The main contribution of this study lies in its integration of drawing tasks with deep learning and natural language processing techniques to quantitatively assess the flexibility and originality aspects of creativity, which allows for rigorous testing of the hypothesized flexibility pathway linking positive mood states to creativity. Specifically, this study will measure flexibility using the Compositional Stroke Embedding (CoSE) model, which utilizes a Gaussian Mixture Model (GMM) to predict potential strokes in the incompleteness shape drawing task. This approach assesses the uncertainty and variability between possible strokes, employing metrics such as entropy and Bhattacharyya distance to capture the dynamic range of creative options. Complementing this, flexibility will be further evaluated through Divergent Semantic Integration (DSI), which employs BERT-generated embeddings to analyze how participants integrate divergent ideas into their narratives. Meanwhile, originality will be evaluated using the Automated Drawing Assessment (AuDrA) model trained with human ratings on the same incompleteness shape drawing task. Together, this proposed study not only aims to elucidate mood-creativity linkage, but also underscores the transformative potential of integrating artificial intelligence into the study of complex human cognitive processes.

\end{document}