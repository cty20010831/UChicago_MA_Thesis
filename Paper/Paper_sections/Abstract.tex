\documentclass[../MA_Thesis.tex]{subfiles}
\renewcommand{\baselinestretch}{1.5} 
\usepackage{hyperref}
\usepackage{csquotes}

\begin{document}
This study integrates visual creative tasks with state-of-the-art computational techniques to examine the flexibility pathway proposed by the dual pathway to creativity model, where positive activation moods enhance the originality aspect of creativity (the uncommonness of ideas, solutions, or products) via increased cognitive flexibility (easy switching between thoughts that promotes the exploration and connection of diverse ideas). Diverging from the predominance of verbal tasks, this study employs an incompleteness drawing task to track both the dynamics of the creative process indicative of cognitive flexibility and the originality of the final completed drawings. Using validated film clips for mood induction, we randomly assigned 90 participants to one of three groups: \textit{High-Arousal Positive Mood}, \textit{Low-Arousal Positive Mood}, and \textit{Neutral Control}. Participants then completed three rounds of the Incomplete Shape Drawing Task (with qualitatively distinct incomplete shapes), followed by a section to provide a written narrative on their thought process. Flexibility was assessed through two modalities: (1) stroke-level drawing behavior using the Compositional Stroke Embedding model to estimate stroke-level entropy and Bhattacharyya distance, analyzed as both average values and inflection proportions throughout the drawing process; and (2) semantic-level flexibility using Divergent Semantic Integration to quantify the integration of conceptually distant ideas in participants’ narratives. Originality was evaluated using AuDrA, a deep learning model trained on human creativity ratings from the same task. The results indicated that while mood induction effectively altered participants’ self-reported arousal and valence, no significant differences in flexibility or originality were observed across mood conditions. However, correlational and mediation analyses suggested two distinct modes of flexibility: persistent exploratory breadth (average entropy and Bhattacharyya distance), and adaptive switching (inflection proportions of entropy and Bhattacharyya distance and average DSI). Crucially, only these adaptive switching flexibility measures were significantly associated with originality, with inflection-based and semantic integration metrics emerging as robust predictors. Together, these results raised caveats on transient mood manipulations in creativity research, while illuminating the value of process-level, adaptive flexibility as a core mechanism underlying originality. Moreover, it demonstrates the utility of integrating drawing-based tasks with machine learning and NLP to capture the dynamic, cross-modal mechanisms underpinning original thought.

\end{document}