\documentclass[../MA_Thesis.tex]{subfiles}
\renewcommand{\baselinestretch}{1.5} 
\usepackage{hyperref}
\usepackage{csquotes}

\begin{document}
This study integrates visual creative tasks with artifical intelligence techniques to study the flexibility pathway proposed by the dual pathway to creativity model. According to this model, positive activating moods enhance originality—the uncommonness of ideas, solutions, or products—by increasing cognitive flexibility, which facilitates the switching between thoughts and supports the exploration and connection of diverse ideas. Diverging from the predominance of verbal tasks, this study employs the Incomplete Shape Drawing task to track both the dynamics of the creative process indicative of cognitive flexibility and the originality of the final completed drawings. Using validated film clips for mood induction, I randomly assigned 90 participants to one of three groups: \textit{High-Arousal Positive Mood}, \textit{Low-Arousal Positive Mood}, and \textit{Neutral Control}. Participants then completed three rounds of the Incomplete Shape Drawing Task (with qualitatively distinct stimuli), each followed by a section to provide a written narrative on their thought process. Flexibility was assessed through two modalities: (1) stroke-level drawing behavior using the Compositional Stroke Embedding model to compute entropy and Bhattacharyya distance; and (2) semantic-level flexibility using Divergent Semantic Integration to quantify the integration of conceptually distant ideas in participants’ narratives. Originality was evaluated using AuDrA, a deep learning model trained on human originality ratings from the same task. The results indicated that while mood induction effectively altered participants’ self-reported arousal and valence, no significant differences in flexibility or originality were observed across mood conditions. However, correlational and multilevel regression analyses suggested two distinct modes of flexibility: persistent exploratory breadth and adaptive switching. Crucially, only adaptive switching flexibility measures significantly predicted originality. Together, these results raised caveats on transient mood manipulations in creativity research, while illuminating the value of process-level, adaptive flexibility as a core mechanism underlying originality. Moreover, it demonstrates the utility of integrating drawing-based tasks with artifical intelligence to capture the dynamic, cross-modal mechanisms underpinning original thoughts.

\end{document}