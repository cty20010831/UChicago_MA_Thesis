\documentclass[../MA_Thesis.tex]{subfiles}
\renewcommand{\baselinestretch}{1.5} 
\usepackage{hyperref}
\usepackage{csquotes}

\begin{document}
Building on significant advancements in affective psychology and creativity research, this study acknowledges the well-documented synergy between mood states and creative cognition. As described in the literature, frameworks categorizing mood along pleasant-unpleasant and activated-deactivated dimensions have elucidated the intricate relationships between mood induction methods and their cognitive consequences (\cite{siedlecka_experimental_2019}). Coupled with the burgeoning exploration of creativity's multifaceted nature---encompassing definitions, underlying cognitive processes, and diverse assessment tasks (\cite{kaufman_cambridge_2010})---this mutual enrichment has paved the way for both theoretical propositions and empirical validations of the mood-creativity linkage.

Echoing \textcite{kaufman_cambridge_2010}’s call for innovative assessment techniques that more accurately capture the dynamics of creative thinking, this study advances the mood-creativity debate by introducing a novel task-measurement integration. Specifically, it combines the open-ended richness of the Incomplete Shape Drawing Task—capable of revealing nuanced cognitive strategies—with state-of-the-art artificial intelligence tools for automated analysis. Unlike previous studies that often relied on static, single-response tasks and subjective ratings, this approach enables multi-trial, process-level assessment of creativity through both visual and semantic modalities, providing a more fine-grained and ecologically valid evaluation of the flexibility pathway in creative cognition. Focusing on \textit{domain-general}, \textit{little-c} creativity, this study aims to capture the nuanced effects of mood on creative processes and, more specifically, the (potential) building effects of positive mood states on thought-action repertoires (\cite{fredrickson_role_2001}). By distinguishing positive mood states varying on the activation dimension---happiness (high activation) and calmness (low activation), this study seeks to scrutinize the hypothesized flexibility pathway (as suggested by \textcite{de_dreu_hedonic_2008}'s dual pathway to creativity model) in which positive activating mood states, rather than positive deactivating ones, predict cognitive flexibility, characterized by employing wide-ranging and comprehensive cognitive categories to form associations. The conducive influence of positive (activating) mood is further hypothesized to enhance the originality aspect of creativity (i.e., the uncommonness of ideas, solutions, or products). Specifically, this study is guided by the following two research questions:  
\begin{enumerate}
    \item \textbf{How do positive moods across the spectrum of activation level, including positive moods with high level of activation (e.g., happiness) and positive moods with low level of activation (e.g., calmness), affect cognitive flexibility during the creative ideation process, respectively?}
    \item \textbf{How does cognitive flexibility during the creative ideation process further influence the originality aspect of creativity in the final product (i.e., whether flexibility mediates the relationship between positive mood and the originality aspect of creativity)?}
\end{enumerate}

To answer these research questions, this study adopts a validated film-based mood induction protocol (\cite{siedlecka_experimental_2019}) for mood induction. I randomly assign participants to one of three mood conditions (high-arousal positive, low-arousal positive, or neutral) to explore the effects of positive mood states, ranging from activating to deactivating, on the flexibility pathway of creativity. To capture the flexibility and originality aspects of creativity, I utilize tasks that not only track the dynamics of creative processes, but also incorporate novel methodologies from generative sketch modeling and NLP to examine the proposed building effects of positive mood states on thought-action repertoires (\cite{fredrickson_role_2001}). 

Following Barbot’s (2018) Multi-Trial Creative Ideation (MTCI) framework that emphasizes a multi-stimuli approach and the evolving dynamics of creative ideation, this study adopts the Incomplete Shape Drawing Task paired with post-hoc narratives. This design captures not only the final creative output but also the cognitive process underlying it. The collection of both behavioral (stroke-level drawing data) and semantic (narrative-based) data therefore captures the flexibility aspect of creativity through two complementary perspectives. First, the Compositional Stroke Embedding (CoSE) model, a generative model trained on large-scale human drawing data, was chosen for its ability to model the probabilistic space of motor decisions in drawing. Leveraging the Gaussian Mixture Model (GMM) to predict next-stroke trajectories in the Incomplete Shape Drawing Task, the CoSE model effectively captures not just what participants draw, but how they navigate the possible outcome space for each additional stroke. This probabilistic framework allows us to infer the degree of uncertainty and divergence in participants’ stroke choices, making it particularly well-suited for quantifying the flexibility aspect of creativity. Specifically, flexibility is quantified via average entropy and Bhattacharyya distance (capturing sustained exploratory breadth), as well as the inflection proportions (capturing adaptive strategy switching) throughout the drawing process ((see Appendix \ref{appendix: entropy} and Appendix \ref{appendix: bhattacharyya distance} for details of their formulas)). Second, Divergent Semantic Integration (DSI) quantifies semantic flexibility by measuring the conceptual distance between elements in participants’ post-drawing narratives (see Appendix \ref{appendix: DSI}) for its formula). This metric leverages contextual word embeddings from BERT, a powerful language model trained on large corpora, to capture nuanced meanings in natural language. By computing the average pairwise distance between sentence embeddings within each narrative, DSI estimates how conceptually diverse or integrated the participant’s ideas are. This approach is well-suited for creativity research because it moves beyond surface-level lexical diversity and instead captures deeper semantic divergence—a key indicator of flexible thinking in ideation. Meanwhile, the originality aspect of creativity (i.e., the originality of the final completed drawings) is evaluated using AuDrA, a deep learning model trained on human-rated drawings from the same task. Mediation analyses are conducted to test whether flexibility mediates the effect of mood condition on originality. Mood condition is treated as a multicategorical independent variable, with the neutral group serving as the reference category. Dummy coding is used to create two binary contrasts: one comparing the high-arousal positive mood group to the neutral control (D1), and another comparing the low-arousal positive mood group to the neutral control (D2). Each flexibility metric is tested as a separate mediator in a parallel mediation framework.

The primary contribution of this study is its integration of drawing tasks with deep learning and natural language processing techniques to quantitatively assess both the flexibility and originality aspects of creativity. This method offers a more comprehensive analysis than previous studies that relied solely on verbal tasks such as the alternative use task and the remote association task (e.g., \cite{kenett_investigating_2014}; \cite{kenett_flexibility_2018}) to quantify the flexibility aspect of creativity. Complementing the predominance of verbal creativity assessments with visual creativity (a canonical form of creative expression; \cite{morrisskay_evolution_2010}) allows for a holistic understanding of creative processes. Verbal and visual creativity engage different cognitive and neural pathways; verbal creativity often relies on language-based processes and abstract thinking (\cite{benedek_create_2014}), while visual creativity engages spatial reasoning and visual-motor coordination (\cite{schlegel_artist_2015}). By studying both aspects, this proposed study could uncover complementary insights into the cognitive mechanisms underlying the flexibility aspect of creativity. This integrative approach enhances the validity of the measurements by capturing a broader range of creative expression and providing a more nuanced understanding of the dynamics of creative thinking. Meanwhile, when it comes to assessing the originality aspect of creativity (especially visual creativity), using automated scoring methods effectively addresses several practical limitations in creativity research. These include the high labor costs associated with manual evaluations and the inherent subjectivity that can bias expert ratings (\cite{patterson_audra_2023}). Automated methods provide a scalable and consistent way to evaluate originality, ensuring that the assessments are objective and replicable across different studies.

\end{document}