\documentclass[../MA_Thesis.tex]{subfiles}
\renewcommand{\baselinestretch}{1.5} 
\usepackage{hyperref}
\usepackage{csquotes}

\begin{document}
Aiming to join the debate on mood-creativity linkage, this study combined visual creative tasks (specifically, the incomplete drawing task) with state-of-the-art computer vision and NLP techniques to capture both the creative processes that demonstrate the flexibility aspect of creativity and the final products of drawings that show the originality aspect of creativity. By successfully manipulating participants’ mood states (particularly along the arousal dimension) into \textit{High-Arousal Positive Mood}, \textit{Low-Arousal Positive Mood}, and \textit{Neutral Control} groups using three validated film clips, the current study examined the flexibility pathway in the dual pathway to creativity model, where positive activation moods enhance cognitive flexibility, thereby increasing originality in creative output.

Despite successful mood manipulation-particularly along the arousal dimension-there were no significant group differences in CoSE-based flexibility measures, semantic integration (DSI), and originality (AuDrA) scores across induced mood conditions. As such, a core precondition for conducting mediation analysis (i.e., a significant relationship between the independent variable and the mediator and/or dependent variable) was not met. Therefore, I decided not to perform the planned mediation analysis and instead shift the focus to evaluate the process-level relationship between flexibility and originality (i.e., the flexibility pathway to creativity according to the dual pathway to creativity model). 

Both correlation and multilevel regression analyses revealed a consistent dissociation between two types of flexibility. Average entropy and average Bhattacharyya distance were highly correlated with one another, suggesting a shared representation of persistent exploratory breadth, but they showed weak associations with originality. In contrast, inflection-based metrics and DSI—indicators of dynamic switching and semantic richness—were positively correlated with one another and significantly predicted originality. These effects were confirmed in a multilevel regression model that accounted for the nested structure of the data and controlled for trait affectivity, openness to experience, cognitive flexibility, and self-rated artistic skill. Three predictors—inflection proportion of entropy, inflection proportion of Bhattacharyya distance, and DSI—emerged as significant contributors to originality, while the average-based measures did not.  

\subsection*{The Elusive Role of Positive Mood in Creativity}
The present study aimed to test a core prediction of the dual pathway to creativity model (\cite{nijstad_dual_2010}), where positive activating moods enhance cognitive flexibility and thus increase creative originality. While the manipulation check confirmed that the mood induction procedure successfully altered both arousal and valence—particularly distinguishing the \textit{High-Arousal Positive Mood} group from the \textit{Neutral Control}—its downstream impact on creative performance was unexpectedly limited.
Across all creativity-related measures (i.e., average entropy, average Bhattacharyya distance, inflection proportion of entropy, inflection proportion of Bhattacharyya distance, DSI, and AuDrA), none showed significant between-group differences in one-way ANOVAs across the three induced mood conditions.

These findings appear to diverge from the general trend reported in previous meta-analyses, which suggest that positive mood enhances creativity by broadening cognitive categories and fostering associative thinking (e.g., \cite{davis_understanding_2009}; \cite{fredrickson_role_2001}). However, a growing body of research has also documented small or nonsignificant effects of positive mood on creativity, pointing to a more complex and context-sensitive relationship. For example, \textcite{baas_meta-analysis_2008} conducted a meta-analysis spanning 25 years of research and found only a modest average effect size of $r = .15$ for the link between positive mood and creativity, with substantial variability between studies. Similarly, \textcite{zenasni_effects_2002} reported that positive mood does not reliably enhance creativity across all contexts, especially when tasks involve higher cognitive demands or when participant characteristics (e.g., openness to experience, emotion regulation) vary widely. These findings suggest that positive mood may not consistently yield measurable gains in creative output, particularly under certain experimental conditions.

In the current study, several aspects of the design may have attenuated the influence of mood on creativity. The limited effect of mood on process-based flexibility may be due to the transient nature of the induced affective states. Prior research has shown that mood manipulations involving brief film clips can quickly fade, especially in online experimental environments where participant engagement and environmental control are inherently limited (\cite{fong_effects_2006}). Additionally, a mismatch in the time scale between mood induction and the drawing task may have further diluted the intended effect. The delay between the end of the film clip and the execution of the creative task could have limited the extent to which the induced mood persisted through the peak period of cognitive demand (\cite{monno_duration_2024}). Future studies could address this issue by borrowing strategies from mental health research, such as incorporating real-time mood sampling, repeated mood inductions, or task paradigms that tightly integrate affective and cognitive processes. Approaches such as gamified interfaces or feedback-sensitive interactive environments, which have been shown to enhance emotional engagement and sustain affective states in mental health interventions (e.g., \cite{balaskas_ecological_2021}), may help preserve induced mood throughout periods of peak creative demand and lead to more robust effects on creative performance.

\subsection*{Two Modes of Creative Flexibility: Persistent vs. Adaptive}
Both the correlation and multilevel regression results revealed a notable divergence among the flexibility measures, pointing to two distinct behavioral profiles. Average-based metrics—including average entropy and Bhattacharyya distance—were highly correlated with each other but showed weak or no associations with originality. In contrast, inflection-based metrics and DSI were moderately correlated with each other and significantly predicted originality. This dissociation suggests that the flexibility metrics captured qualitatively different modes of behavior, motivating a closer theoretical examination of what these modes may represent in the context of creative performance.

The first mode, indexed by average entropy and average Bhattacharyya distance, may reflect a persistent exploratory approach, where participants maintained a broad search space throughout the drawing process without frequent shifts in strategy. The second mode, captured by inflection-based metrics and DSI, reflects adaptive flexibility, characterized by frequent changes in drawing direction and greater conceptual integration across narrative elements.

Importantly, while both modes may reflect distinct cognitive strategies, only adaptive flexibility—not persistent exploratory breadth—was reliably associated with higher originality. Inflection-based metrics and DSI showed significant positive relationships with originality (all $p < .001$), whereas average-based metrics did not. This divergence supports a growing literature emphasizing that moment-to-moment switching between ideas, rather than merely sustaining a wide exploratory range, may be more critical to creative performance.

This distinction mirrors dual-process models in creativity and executive function research. For example, \textcite{sowden_shifting_2015} described creativity as involving a flexible interplay between generative and evaluative modes, which may require shifting between cognitive sets. Similarly, \textcite{nusbaum_emily_c_and_silvia_paul_j_individual_2019} argued that executive control plays a critical role in enabling idea switching and overcoming cognitive fixation. The present findings align with these accounts, as participants who frequently switched strategies—captured by drawing inflections—also produced more semantically diverse narratives, as measured by DSI.

The observed pattern also resonates with the work on the exploration-exploitation tradeoff in idea generation. \textcite{hills_exploration_2015} proposed that successful creative thinkers must balance the exploration of new conceptual territories with the exploitation of promising ideas. In this framework, persistent flexibility may reflect broad but undirected exploration, while adaptive flexibility—marked by targeted switching—may be more efficient for generating novel combinations. Notably, the task design in this study (incomplete drawing with open-ended narrative elaboration) may have favored participants who could dynamically adapt to shifting constraints and integrate visual and conceptual modes of ideation.

This distinction can also be mapped to separate brain systems involved in creative thought. \textcite{beaty_creative_2016} suggested that spontaneous and controlled creativity involves distinct but interacting networks: the default mode network supports associative generation, while executive control networks facilitate goal-directed idea evaluation and refinement. Participants who demonstrated adaptive switching may have engaged this interplay more effectively, translating flexibility into higher originality.

Taken together, these findings provide converging support for the idea that not all forms of flexibility are equally predictive of creativity. While persistent exploration may signal openness or fluency, it is the capacity to adaptively shift strategies and integrate across representational modes that appears to drive original output in this task. Future research might examine whether this distinction holds across other creative domains, and whether training interventions can selectively enhance adaptive flexibility as a means of boosting creative performance.

\subsection*{Linking Drawing Dynamics, Semantic Integration, and Originality}
Beyond exploring the effects of (positive) mood states, this study provides important insights into the cognitive pathways linking flexibility and creative originality. Across both drawing-based and narrative-based measures, consistent associations emerged between process-level flexibility and originality, as measured by the AuDrA model. Specifically, the inflection proportions of entropy and the inflection proportion of the Bhattacharyya distance (stroke-level metrics that capture moment-to-moment shifts in drawing behavior) were significant predictors of originality. Similarly, DSI (a semantic measure of narrative flexibility) showed a robust positive association with originality. In contrast, average entropy and average Bhattacharyya distance, which index the sustained breadth of exploration, were not significantly related to originality. 

Although the present study did not replicate the hypothesized effect of positive activating mood on creativity, the findings nevertheless lend support to the flexibility pathway within the dual pathway to creativity model, which posits that creative ideation can arise through adaptive switching between categories or perspectives (\cite{nijstad_dual_2010}). The observed associations between inflection-based metrics and originality support the notion that adaptive shifts in strategy, whether visual or conceptual, are predictive of original creative output. This interpretation is consistent with long-standing theoretical accounts that position ideational flexibility as a core component of creativity. For example, \textcite{guilford_nature_1967} emphasized flexibility as distinct from fluency and originality, highlighting its role in allowing individuals to shift between conceptual categories. Similarly, \textcite{johnson-laird_freedom_1988} described creativity as arising from the generation and transformation of mental models, a process that inherently requires the ability to abandon habitual associations. More recent empirical studies have corroborated this view, showing that flexibility, especially when measured as the ability to switch between semantic fields or representational frames, predicts performance on divergent thinking tasks (\cite{benedek_differential_2012}; \cite{kenett_structure_2016}). There is also converging neuroimaging evidence supporting this flexibility–originality pathway. Building on behavioral findings, research by \textcite{beaty_creative_2016} demonstrates that cognitive flexibility engages dynamic interactions between large-scale brain networks, particularly the default mode network (DMN)—which facilitates associative and spontaneous thought—and the executive control network (ECN)—which enables goal-directed regulation and cognitive control. The inflection-based stroke metrics used in the present study may reflect these underlying neural dynamics, capturing real-time behavioral expressions of transitions between exploratory and evaluative modes of thinking. That DSI, a measure grounded in semantic processing and entirely distinct from motor behavior, also significantly predicted originality further reinforces the idea that adaptive switching, whether expressed visually or conceptually, is a core process underlying creative output.

Furthermore, from a methodological standpoint, the convergence of flexibility-originality pathway across the two modalities (i.e., drawing and narrative) provides empirical support for the validity of the measures employed. Specifically, CoSE-derived inflection metrics successfully captured behavioral flexibility at the stroke level, while DSI effectively indexed semantic flexibility in participants’ post-task narratives. Additionally, the positive association between stroke-based and narrative flexibility measures and the automated rating of drawing originality (i.e., AuDrA) supports the theoretical claim that adaptive, real-time flexibility underlies creative output. Inflection-based metrics captured shifts in visual strategy during the drawing task, while DSI reflected semantic divergence in narrative responses. Their shared ability to predict originality demonstrates convergent validity with AuDrA—a model trained on human ratings—suggesting that both measures capture core aspects of creativity recognized in human evaluations. Meanwhile, the lack of association between average-based metrics and originality, meanwhile, highlights the specificity of the inflection-based measures in capturing meaningful creative dynamics. Taken together, these findings validate the use of automated, quantitative approaches for assessing creativity, particularly in the visual domain where human scoring is traditionally labor-intensive and subjective.

These findings also align with a broader methodological shift in the cognitive sciences: the increasing use of cross-modal and machine learning approaches to investigate higher-order cognitive processes such as attention, memory, and executive function. By integrating data across sensory modalities and leveraging computational models to decode brain or behavioral signals, these approaches allow researchers to capture the complexity and dynamism of cognition with greater granularity than traditional methods (e.g., \cite{ye_self-supervised_2024}; \cite{vessel_default-mode_2019}). This paradigm has also begun to influence creativity research, where the need to assess process-level dynamics in multiple representational domains—such as vision, semantics, and motor—has grown. In particular, drawing-based tasks, once primarily assessed through subjective scoring, are increasingly being paired with computational tools to extract features such as semantic divergence or stroke dynamics. These developments enable more scalable and interpretable models of creativity, and the present study contributes to this growing framework by demonstrating how both visual and narrative flexibility can be quantitatively linked to model-assessed originality. Recent advances in sketch-to-image translation (e.g., \cite{ghosh_interactive_2019}; \cite{wang_diffsketching_2023}) demonstrate how neural networks can interpret sparse visual input and generate semantically rich visual outputs, reflecting the generative potential of sketch-based expression. Meanwhile, work at the intersection of text and drawing has explored how multimodal inputs, such as hand-drawn sketches combined with textual prompts, can guide complex generative systems (\cite{chen_control3d_2023}), offering new possibilities for modeling creativity in naturalistic and user-directed settings. The present study contributes to this growing body of research by showing that stroke-based and semantic flexibility measures, both derived from a drawing-based task, meaningfully predict originality as assessed by a computational model (AuDrA). These results suggest that drawing, when paired with principled computational methods, is a powerful and underutilized modality for assessing creative processes across cognitive and representational domains.

\subsection*{Strengths, Limitations and Future Directions}
Aiming to investigate the mood-creativity linkage (particularly the flexibility pathway) as proposed in the dual pathway to creativity model, this study offers several key contributions to creativity research. First, it provides empirical evidence for two distinct modes of creative flexibility: (1) a persistent exploratory approach, characterized by sustained entropy and broad stroke divergence, and (2) an adaptive flexibility, marked by real-time shifts in drawing strategy and semantic integration. Crucially, only the adaptive mode of flexibility—captured through inflection-based stroke dynamics and DSI—predicted originality, supporting the theoretical claim that the ability to shift between ideas, rather than simply maintaining a broad search space, underlies creative ideation. Second, by combining stroke-based metrics derived from drawing behavior and language-based metrics extracted from narrative responses, the study demonstrates a cross-modal approach to measuring cognitive flexibility. This integration captures flexibility at both the sensorimotor level (via shifts in visual-motor execution) and the conceptual level (via semantic divergence in language), offering converging evidence for the underlying cognitive processes that contribute to originality.
Third, the study employs an ecologically valid low-instruction drawing task that allows participants to generate creative responses with minimal constraints. Unlike traditional divergent thinking tasks, which often rely on highly structured prompts and focus predominately on the verbal domain, this approach supports spontaneous and context-sensitive creative expression. By embedding creativity assessment in a simple open-ended visual completion task, the study captures creativity as it unfolds more naturally, aligning more closely with real-world creative behavior and reducing cultural or linguistic bias that may affect verbal tasks. Finally, this study introduces a scalable and automated framework for creative assessment that significantly reduces the dependency on labor-intensive human scoring. By leveraging machine learning and natural language processing techniques—including the CoSE model for drawing analysis, DSI for narrative analysis, and AuDrA for originality evaluation—the framework supports efficient, reproducible, and interpretable assessments of creativity across multiple modalities.

Despite boasting these strengths, this study is not without limitations, which point toward promising directions for future research. First, although the sample size ($N = 90$) was sufficient to detect medium-to-large effects, it may have been underpowered to capture the subtler effects of mood on flexibility and originality, which are often modest in size. Meta-analytic evidence suggests that mood influence on creativity tends to be small and highly context dependent (\cite{baas_meta-analysis_2008}), and detecting such effects may require larger samples, especially when examining indirect pathways or interactions. Moreover, the transient nature of mood states induced through brief film clips raises questions about the duration and depth of emotional engagement during the task. Future research could use repeated or sustained mood manipulations, mood tracking across the task, or immersive delivery methods (e.g., virtual reality) to better capture the dynamic relationship between affect and creative processing. Second, despite improving accessibility and ecological reach, the online experiment in this study also introduces uncontrolled variability in participant environments. Factors such as external distractions and reduced engagement may have weakened the effects of mood or interfered with task performance. Although basic quality control procedures were used, future studies could incorporate richer behavioral indicators of engagement—such as drawing latency, cursor dynamics, or time-on-task measures—to assess and filter data quality more systematically. Third, the study focused solely on behavioral and narrative data without probing the neurocognitive mechanisms underling flexibility and originality. Previous research has emphasized the importance of dynamic interactions between the default mode network (DMN) and the executive control network (ECN) in supporting creative thought (e.g., \cite{beaty_creative_2016}). To more directly examine the neural correlates of adaptive flexibility, future studies could incorporate neurophysiological measures such as EEG or fMRI. For instance, EEG-based time-frequency analyses could be used to track changes in frontal midline theta activity, which has been linked to cognitive control and set shifting. Similarly, event-related potentials (ERPs) could isolate neural responses to inflection points in the drawing task, marking moments of strategic switching. Furthermore, using fMRI, researchers could apply functional connectivity analyses or dynamic causal modeling (DCM) to examine how activation and communication between DMN and ECN evolve throughout the creative process. Combining these neural indices with stroke- or narrative-level flexibility metrics would allow for a more integrated, multimodal understanding of how brain dynamics support creative behavior in real time. Fourth, the analysis pipeline relied on summary-level metrics (e.g., entropy, inflection proportion), which, while interpretable and efficient, may oversimplify the temporal structure of the creative process. Drawing is inherently sequential and dynamic, offering opportunities to model ideation and strategy shifts over time. Future work could adopt more process-sensitive modeling approaches, such as Hidden Markov Models (HMMs), Bayesian switching models, or dynamic time warping, to identify latent cognitive states or transitions in creative behavior. These methods would allow researchers to move beyond static indicators and toward a more process-oriented account of flexibility. Finally, although the AuDrA model provides a scalable, automated measure of originality, the current study did not include human ratings of creativity for comparison. Incorporating dual evaluation methods (i.e., combining human ratings with algorithmic scores) could improve the validity of machine-based creativity assessments and may reveal meaningful divergences between human and computational perspectives on originality.

\end{document}