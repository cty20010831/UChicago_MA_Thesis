% Choose whether for proposal or writing sample.
\documentclass[../Proposal_Writing_Sample.tex]{subfiles}
% \documentclass[../Proposal_MA_Thesis.tex]{subfiles}
\renewcommand{\baselinestretch}{1.5} 
\usepackage{hyperref}
\usepackage{csquotes}

\begin{document}
I would like to express my gratitude to Dan and Dr.~Pedro for their constructive feedback on the first draft of my proposal. Their insightful comments and suggestions have been instrumental in refining my research direction and enhancing the overall quality of my work. 

Dan has highlighted several critical areas for improvement, including highlighting literature gaps, placement of figures, reference sources, feasibility concerns, consistency of concept usage, and repetitive information. First, I acknowledged that the literature gap for measuring the flexibility aspect of creativity is not well-introduced. Hence, I revised the part on the contribution of my methodological approaches and research design by highlighting what I believe were the areas of improvement from previous studies (i.e., creativity assessment). Second, Dan correctly identified the ill positioning of one of my figures (specifically, Figure 4), which showed until page 13 despite being mentioned on page 11. Following his advice, I have updated my latex code to fix this issue. Third, Dan pointed out a mistake I made when arguing that ``experimental mood induction has been shown to be effective in altering mood and provides strong causal evidence of its effects on creativity (Westermann et al., 1996).'' Specifically, \textcite{westermann_relative_1996}'s paper does not discuss the causal evidence of mood on creativity. Following Dan's suggestion, the in-text citation block was placed immediately after ``Experimental mood induction has been shown to be effective in altering mood.'' Fourth, Dan correctly pointed out that I mistakenly used the words \textit{emotion} and \textit{mood} interchangeably. I corrected these errors to avoid confusion, especially after highlighting the difference between \textit{mood} and \textit{emotion} in my proposal. I also followed Dan's advice to use only the word \textit{activation} (instead of interchangeably using \textit{activation} and \textit{arousal}) for clarity. Fifth, I followed Dan’s advice to revise the part (originally) on page 17, where I provided repetitive information on the AuDrA model. Finally, I would like to thank Dan for his suggestion to include a circumplex model as a complementary visual guide while introducing the dual dimensions of mood.

Dr.~Pedro also pinpointed two invaluable pieces of advice. He first suggested that I should cite more of the relevant literature and information in the opening section to provide a solid foundation and context for my research. Following his advice, I included more references covering creativity as a fascinating human capacity, the standard definition of creativity, the multidimensional construct of creativity, fundamental cognitive ability, the influence of mood on cognitive processes, and the relationship between cognitive flexibility and creativity. In total,  I added five additional references from the literature on mood and creativity to the opening section of my proposal, which helps situate my study within the existing body of work and highlight its significance. Dr.~Pedro also recommended that I develop a stronger argument for why my methodological approach is a significant contribution to the field. Following his suggestion, I extended my previous section on the contribution of my proposed study by more explicitly discussing why the integrative approach to assessing both verbal and visual creativity captures a broader range of creative expression and provides deeper insights into the cognitive mechanisms underlying the flexibility aspect of creativity. In addition, I highlighted the importance of using an automated scoring of originality via deep learning methods, as it reduces the costs and increases the objectivity and scalability of the results. 

In conclusion, the feedback from Dan and Dr.~Pedro has been invaluable in shaping and refining my proposal. Their detailed reviews have not only highlighted areas of improvement, but have also provided practical solutions to enhance the clarity, rigor, and impact of my research.

\vspace{1cm}
\begin{flushright}
Sincerely,\\
Sam Cong
\end{flushright}

\end{document}