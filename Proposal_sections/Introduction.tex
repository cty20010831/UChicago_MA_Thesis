\documentclass[../Proposal.tex]{subfiles}
\renewcommand{\baselinestretch}{1.5} 
\usepackage{hyperref}
\usepackage[style = apa]{biblatex}
   \addbibresource{references.bib}
\usepackage{csquotes} % Package for direct quotation
\usepackage{placeins} 
\usepackage{geometry} 
    \geometry{margin=1in} 
\usepackage{amsmath}
\usepackage{graphicx}
\usepackage{indentfirst} % Make sure the first paragraph after a section heading is indented

\begin{document}

\begin{center}
\textit{``There is no doubt that creativity is the most important human resource of all. Without creativity, there would be no progress, and we would be forever repeating the same patterns.''} 
\end{center}
\begin{flushright}
-- Edward De Bono
\end{flushright}

Among the multiple strands of research exploring the content of cognition and the cognition processes that make humans unique, creativity stands out as a fascinating human capacity to formulate novel ideas, methods, and solutions (\cite{hennessey_creativity_2010}). While creativity can manifest as ``big C'' creativity, involving major breakthroughs that propel our civilization forward, it can also appear as ``little c'' creativity, which helps solve myriad of everyday problems through routine creative acts (\cite{nijstad_dual_2010}; \cite{richards_everyday_2007}). Following Guilford's famous presidential address to the American Psychological Association where he pinpointed the lack of research on creativity in 1950 (\cite{de_alencar_theory_2021}; \cite{gaut_philosophy_2010}), the field of creativity has witnessed burgeoning development leading to the influential \textit{standard} definition of creativity: ``creativity is usually defined as the generation of ideas, insights, or problem solutions that are new and meant to be useful'' (\cite[739]{de_dreu_hedonic_2008}). However, this standard definition does not overshadow the multiple dimensions of creativity, nor does it confine creativity studies to a homogeneous set of theories explaining the nature and process of creativity. Instead, creativity is increasingly recognized as a multidimensional construct that incorporates various facets, including \textit{cognitive}, \textit{personal}, \textit{developmental}, and \textit{social} factors (\cite{kaufman_cambridge_2010}; \cite{plucker_why_2004}; \cite{simonton_creativity_2000}). Notably, theories of cognitive psychology illuminate that creativity is not merely an isolated trait of exceptionally gifted people, but rather a fundamental cognitive ability inherent in all individuals, characterized by complex cognitive processes such as the generation of novel ideas, the recombination of existing information, and the redefinition of problems from new perspectives (\cite{finke_creative_1996}; \cite{ward_conceptual_1997}). 

An essential component of these cognitive processes is the influence of mood. As diffuse affective states that are not targeted at any particular object (\cite{desmet_15_2008}), mood pervades our entire framework of meaning and shapes our perception of the possibilities that the world offers (\cite{ratcliffe_why_2013}). Unlike emotions, which are acute and directed responses, moods are diffuse and enduring affective states that subtly color our psychological landscape (\cite{lischetzke_mood_2022}). Importantly, mood significantly affects cognition by influencing what we think and the efficiency of our cognitive processes, which is supported by overlapping neural networks between mood and cognition, as identified in functional neuroanatomy studies (\cite{chepenik_influence_2007};\cite{dolcos_neural_2011}; \cite{storbeck_interdependence_2007}). These studies reveal that both cortical and subcortical brain regions are involved, suggesting a deeply integrated system where mood can influence various cognitive functions, including perception, attention, and memory. Evidence from both functional neuroimaging and behavioral studies underscores that the substrates of cognition are not only shared with but also significantly influenced by mood states. This includes direct effects observed in studies involving individuals with psychiatric disorders and experimental studies in which mood states are induced in healthy participants to assess cognitive impacts (e.g., \cite{cabeza_imaging_2000}; \cite{iosifescu_relation_2012}; \cite{phan_functional_2004}). These findings collectively suggest a powerful interplay between mood and cognition that can either facilitate or hinder cognitive processes depending on the nature of mood involved. 

The broaden-and-build theory proposed by \textcite{fredrickson_role_2001} complements this understanding by suggesting that positive emotions specifically \textit{broaden} an individual's thought-action repertoire, including an expanded locus of attention and increased focus on the big picture of the situation. Relating it back to creativity, which involves similar components of allocating attention resources and adjusting processing styles, the broadened cognitive scope could reasonably enable individuals to form more novel combinations and see connections between disparate ideas. This cognitive flexibility is essential for creative thinking, as it allows for more complex, abstract, and innovative thinking processes (\cite{isen_positive_1987}). Meanwhile, it is also worth considering whether different \textit{activation} levels of positive mood states unequivocally facilitate creative thinking. Together, this leads us to the pivotal question: \textbf{How would positive moods influence creativity?} This inquiry not only probes the capacity (and perhaps boundary conditions) of positive mood states to enhance creative output, but also explores the mechanisms through which mood may dynamically interact with the cognitive processes essential for creative thinking.

To answer this question, this study refers to the dual pathway to creativity model (\cite{de_dreu_hedonic_2008}) to examine the link between positive mood and creativity. Specifically, it integrates the incompleteness drawing task with state-of-the-art deep learning and natural language processing (NLP) techniques to quantitatively assess the flexibility and originality aspects of creativity, enabling the empirical falsification of the flexibility pathway in the dual pathway to creativity model.

\end{document}